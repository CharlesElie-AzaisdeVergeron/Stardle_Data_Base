\documentclass{article}
\usepackage{graphicx} % Required for inserting images
\usepackage{hyperref}
\usepackage[ruled,vlined]{algorithm2e}
\usepackage{pythontex} 
\usepackage{listings}
\usepackage{xcolor}

% Define a nice style for Python code
\lstdefinestyle{pythonstyle}{
    language=Python,
    basicstyle=\ttfamily\small,
    breaklines=true,
    commentstyle=\color{green!60!black},
    keywordstyle=\color{blue},
    stringstyle=\color{red},
    numbers=left,
    numberstyle=\tiny\color{gray},
    backgroundcolor=\color{gray!10},
    frame=single
}



\title{Compte-rendu du projet d'analyse de données: applications au jeu Stardle}
\author{Charles Azais de Vergeron et Octave Feuilland }
\date{Mai 2025}

\begin{document}

\maketitle

\section{Introduction}
\subsection{Présentation du jeu Stardle}

Le jeu Stardle est un jeu en ligne (\href{https://stardle.net}{stardle.net}) basé sur le jeu Wordle / SUTO. 
Le but est de deviner chaque jour un vaisseau choisi au hasard parmi une liste  d'environ 200 éléments. 

Chaque vaisseau est definit par deux types de caractéristiques: 
\begin{itemize}
    \item Qualitatives
    \begin{itemize}
        \item Fabricant (Manufacturer)
        \item Type
        \item Role 
        \item Année d'apparition (Relese Date)
        \item En Jeu ou Non (Status)
    \end{itemize}
    \item Quantitatives
    \begin{itemize}
        \item Nombre de personnes dans l'équipage (Crew)
        \item Valeur en \$ (Price)
        \item Valeurs en monnaie du jeu (Price In game)
        \item Capacité en tant que cargo (Cargo Capacity)
        \item Vitesse de croisière (SCM) et maximale (Max)
        \item Longeur (Length), Largeur (Beam) et Hauteur (Height)
    \end{itemize}
\end{itemize}

Voici un exemple de partie du jeu : \\


Nous avons 3 possibilités de résultats: Vrai, Faux ou Proche (uniquement pour les caractéristiques quantitatives).
\subsection{Objectif du projet}

L'objectif de ce projet est d'analyser les données du jeu Stardle afin de créer un algorithme capable de trouver 
un vaisseau pris au hasard dans la base de données en un minimum de tentatives et de déterminer une combinaison 
d'essais de vaisseaux qui permet de minimiser le nombre de tentatives.

Nous avons donc récupérer les données et recréer le jeu puis essayer de trouver un 
algorithme optimal pour déterminer le vaisseau le plus efficacement possible. 

\section{Récupération des données}
La récupération des données est faite via le script Python CreateShipDatabase.py. 
Ce script utilise l'API officielle de \href{https://starcitizen.tools/}{Star Citizen Wiki}  pour récupérer les informations sur les vaisseaux.

\subsection{Structure du script}
Le script se décompose en plusieurs étapes principales :

\begin{enumerate}   
    \item \textbf{Collecte des données} \\
        Nous utilisation l'API \verb|https://api.star-citizen.wiki/| pour obtenir la liste complète des 
        vaisseaux.
        Pour chaque vaisseau, le script récupère les données via l'API et 
        extrait toutes les caractéristiques -- dans la configuration "All" -- 
        et uniquement celles du jeu Stardle -- dans la configuration  "Stardle" --\\
        Dans la suite du rapport, "Stardle" désigne la configuration du jeu et "All" la configuration complète.
    \item \textbf{Stockage des données}
    
    \begin{itemize}
        \item Création de deux fichiers JSON :
        \begin{itemize}
            \item \verb|shipList.json| : liste des vaisseaux et leurs liens
            \item \verb|shipDB_All.json| : base de données complète des vaisseaux
            \item \verb|shipDB_Stardle.json| : base de données avec les catégories du jeu Stardle 
        \end{itemize}
    \end{itemize}
\end{enumerate}


\section{Nettoyage et préparation des données}

Pour ajouter de la complexité aux données d'origine, nous avons mélangés les deux base de données.
Stardle est la base du projet. Nous avons extrait de All les valeurs de scm,max, length, beam. Nous avons aussi 
rajouté une colonne "price in game" qui correspond à la valeur du vaisseau dans le jeu (via le site 
\href{https://www.erkul.games/live/calculator}{Erkul.com})\\

Pour la gestion des valeurs manquantes, nous avons étudié les dependances entre les colonnes.
Nous avons remarqué que certaines colonnes étaient très corrélées entre elles. Par exemple, le prix \$ (sans valeurs manquantes) 
et celle du prix en jeu (avec beaucoup des valeurs manquantes). Nous avons tenté de remplr les valeurs manquantes
en utilisant une regression lineaire et une regression quadratique mais ce ne fut pas concluant. 
Nous avons donc appliqué l'algorithme suivant: \\
\\

\begin{algorithm}[H]
    \SetAlgoLined
    \KwResult{Colonne sans valeurs manquantes}
    groupby by status,type,manufacturer\;
    \For{vaisseaux in names}{
        \eIf{mean(groupby by status,type,manufacturer) $\neq$ Nan \textbf{and} price\_in\_game == 0}{
            price\_in\_game[vaisseaux] = mean(groupby by status,type,manufacturer)\;
        }{
            price\_in\_game[vaisseaux] = mean\;
        }
    }
\end{algorithm}


\section{Implementation du jeu}

Nous avons recréé le jeu Stardle en Python : \verb|stardle.py|
Nous avons commencé par afficher l'intégralité des vaisseux du jeu pour éviter au joueur de se tromper
lors de la saisie.
Nous reproduisons la meme logique que le jeu original
puis après nous avons à taper le nom du vaisseux. Ca affiche les résultats.

Voici ce que donne le début du jeu : \\

INSERT CODE 



Il est possible de faire défiler les noms des 200 vaisseaux

Ensuite, tout en bas, il y a une barre de recherche où l'on écrit le nom du vaisseau que nous voulons 
proposer. 

Voici le résultat que ça donne quand nous nous trompons de vaisseau :

\begin{verbatim}
    Vous vous êtes trompés de vaisseau. Voici les résultats de chaque variables :
     -pour la variable cargo_capacity ce n est pas la bonne réponse X
     -pour la variable mass ce n est pas la bonne réponse X
     -pour la variable crew ce n est pas la bonne réponse X
     -pour la variable manufacturer ce n est pas la bonne réponse X
     -pour la variable type c est la bonne réponse V
     -pour la variable status ce n'est pas la bonne réponse X
     -pour la variable role c est la bonne réponse V
     -pour la variable price ce n est pas la bonne réponse   X
     -pour la variable release_date ce n est pas la bonne réponse  X
     -pour la variable price_ingame ce n est pas la bonne réponse  X
     -pour la variable scm ce n est pas la bonne réponse    X
     -pour la variable max ce n est pas la bonne réponse    X
     -pour la variable length ce n est pas la bonne réponse X
     -pour la variable beam ce n est pas la bonne réponse   X
     -pour la variable height ce n est pas la bonne réponse X

\end{verbatim}

Nous pouvons y constater que chaque caractéristique du vaisseau est comparée à celle du vaisseau mystère, pour donnée des indications.
Dans le cas d'un succès, on félicite le joueur pour son résultat comme ci-dessous : 
\begin{verbatim}
    Bravo, vous avez trouvez le bon vaisseau!
    Félicitations! Vous avez trouvé le vaisseau!    
\end{verbatim}


ON TRAITE L ANNEE DE SORTIE ET LE NBRE D EQUIPAGE COMME DU QUALITATIF PCQ C EST DES ENTIERS ET PAS LISSE DU TT 
METTRE UN GRAPHIQUE QUI MONTRE QUE LE NOMBRE D EQUIPAGE ET LA DATE DE SORTIE VARIE PEU SUR TT LES DONNEES
\section{Algorithme de jeu}

Dans un premier temps, nous avons étudié les réparitions des vaisseaux entre differents critères qualitaitfs
par des histogrammes et des graphes. Les histogrammes précisent la répartition des vaisseaux entre leurs roles, fabricants..
alors que des graphes précise les relations entre eux.




\subsection{Algorithme de récupération}
\begin{algorithm}[H]
\SetAlgoLined
\KwData{URL de l'API Star Citizen}
\KwResult{Deux fichiers JSON (shipList.json et shipDB.json)}

Initialiser ShipName = \{\} et ShipDB = \{\}\;
Requête GET à l'API pour obtenir la liste des vaisseaux\;
\For{chaque vaisseau dans la liste}{
    Requête GET pour obtenir les détails du vaisseau\;
    \eIf{DataBaseType == "Stardle"}{
        Extraire: nom, fabricant, rôle, longueur, prix, cargo, équipage\;
        Ajouter au ShipDB\;
    }{
        Sauvegarder toutes les données du vaisseau\;
    }
}
Sauvegarder ShipName dans shipList.json\;
Sauvegarder ShipDB dans shipDB\_[Type].json\;
\caption{Récupération des données des vaisseaux}
\end{algorithm}


\end{document}